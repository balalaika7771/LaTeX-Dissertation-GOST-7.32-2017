% Термины и определения (по усмотрению исполнителя НИР)
\chapter*{ТЕРМИНЫ И ОПРЕДЕЛЕНИЯ}
\addcontentsline{toc}{chapter}{ТЕРМИНЫ И ОПРЕДЕЛЕНИЯ}

В настоящем отчете о НИР применяют следующие термины с соответствующими определениями:

\begin{description}
    \item[Автоматизированная информационная система] — комплекс программных, технических, информационных, лингвистических, организационно-технологических средств и персонала, предназначенный для сбора, формирования, хранения, обработки, поиска и распространения информации.
    
    \item[База данных] — совокупность данных, организованных в соответствии с концептуальной схемой, описывающей характеристики этих данных и взаимоотношения между ними.
    
    \item[Информационная система] — организационно упорядоченная совокупность документов, информационных технологий и реализующих их технических средств.
    
    \item[Маркетинг] — вид человеческой деятельности, направленный на удовлетворение нужд и потребностей посредством обмена.
    
    \item[Маркетинговое планирование] — процесс разработки планов маркетинговой деятельности предприятия на определенный период времени.
    
    \item[Методология] — система принципов и способов организации и построения теоретической и практической деятельности.
    
    \item[Научно-исследовательская работа] — деятельность, направленная на получение новых знаний о природе, обществе и мышлении.
    
    \item[Объект исследования] — явление или процесс, на который направлена познавательная деятельность субъекта.
    
    \item[Предмет исследования] — стороны, свойства, отношения объекта, изучаемые в данной работе.
    
    \item[Цель исследования] — желаемый результат научной деятельности, выраженный в качественных и количественных показателях.
\end{description}

\newpage
