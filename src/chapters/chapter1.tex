\chapter{Теоретические основы автоматизации производственных процессов}

\section{Основные понятия и определения}

В данной главе рассматриваются теоретические основы систем автоматизации производственных процессов. В качестве теоретической основы использованы классические работы по программированию \cite{example_book} и современные исследования в области машинного обучения \cite{ml_conference}.

Автоматизация производственных процессов представляет собой комплекс технических средств и методов, направленных на повышение эффективности производства за счет уменьшения участия человека в технологических процессах.

\subsection{Ключевые термины автоматизации}

Ключевые термины и их определения представлены в таблице~\ref{tab:automation_terms}.

\begin{table}[H]
\centering
\caption{Основные термины автоматизации}
\label{tab:automation_terms}
\begin{tabular}{|p{4cm}|p{8cm}|}
\hline
\textbf{Термин} & \textbf{Определение} \\
\hline
Автоматизация & Процесс замены ручного труда машинным \\
\hline
Система управления & Совокупность устройств для управления технологическими процессами \\
\hline
Промышленный контроллер & Устройство для автоматического управления оборудованием \\
\hline
SCADA-система & Система диспетчерского управления и сбора данных \\
\hline
HMI & Человеко-машинный интерфейс \\
\hline
\end{tabular}
\end{table}

\subsection{Архитектура системы автоматизации}

Современные системы автоматизации имеют многоуровневую архитектуру, представленную на рисунке~\ref{fig:automation_architecture}.

\begin{figure}[H]
\centering
\begin{tikzpicture}[node distance=2cm]
    % Уровень управления предприятием
    \node[rectangle, draw, fill=blue!20, minimum width=3cm, minimum height=1cm] (erp) {ERP-система};
    
    % Уровень управления производством
    \node[rectangle, draw, fill=green!20, minimum width=3cm, minimum height=1cm, below of=erp] (mes) {MES-система};
    
    % Уровень управления процессом
    \node[rectangle, draw, fill=yellow!20, minimum width=3cm, minimum height=1cm, below of=mes] (scada) {SCADA};
    
    % Уровень управления оборудованием
    \node[rectangle, draw, fill=red!20, minimum width=3cm, minimum height=1cm, below of=scada] (plc) {ПЛК};
    
    % Связи между уровнями
    \draw[->] (erp) -- (mes);
    \draw[->] (mes) -- (scada);
    \draw[->] (scada) -- (plc);
\end{tikzpicture}
\caption{Архитектура системы автоматизации}
\label{fig:automation_architecture}
\end{figure}

На рисунке \ref{fig:automation_architecture} представлена иерархическая структура системы автоматизации, включающая четыре основных уровня управления.

\subsection{Математические основы управления}

Основное уравнение динамики системы автоматизации:

\begin{equation}
\frac{dx(t)}{dt} = Ax(t) + Bu(t) + w(t)
\label{eq:system_dynamics}
\end{equation}

где $x(t)$ — вектор состояния системы, $u(t)$ — вектор управления, $A$ и $B$ — матрицы системы, $w(t)$ — возмущение.

Для оценки качества управления используется квадратичный критерий:

\begin{equation}
J = \int_0^T [x^T(t)Qx(t) + u^T(t)Ru(t)] dt
\label{eq:performance_criterion}
\end{equation}

где $Q$ и $R$ — весовые матрицы.

\begin{figure}[H]
\centering
\begin{tikzpicture}[scale=0.8]
\begin{axis}[
    xlabel={Время, с},
    ylabel={Температура, °C},
    title={Реакция системы на ступенчатое воздействие},
    grid=major,
    legend pos=north east,
]
\addplot[blue,thick] coordinates {
    (0, 20) (1, 25) (2, 35) (3, 45) (4, 52) (5, 58) (6, 62) (7, 65) (8, 67) (9, 68) (10, 68)
};
\addplot[red,dashed,thick] coordinates {
    (0, 20) (10, 70)
};
\legend{Фактическая температура, Заданное значение}
\end{axis}
\end{tikzpicture}
\caption{Переходный процесс системы управления температурой}
\label{fig:temperature_response}
\end{figure}

\subsection{Сравнительный анализ методов управления}

Результаты сравнительного анализа различных методов управления представлены в таблице~\ref{tab:control_methods_comparison}.

\begin{table}[H]
\centering
\caption{Сравнение методов управления}
\label{tab:control_methods_comparison}
\begin{tabular}{|l|c|c|c|c|}
\hline
\textbf{Метод} & \textbf{Точность, \%} & \textbf{Время отклика, с} & \textbf{Перерегулирование, \%} & \textbf{Сложность} \\
\hline
П-регулятор & 85 & 1.2 & 15 & Низкая \\
\hline
ПИ-регулятор & 92 & 2.1 & 8 & Средняя \\
\hline
ПИД-регулятор & 95 & 1.8 & 5 & Высокая \\
\hline
Адаптивный & 98 & 1.5 & 3 & Очень высокая \\
\hline
\end{tabular}
\end{table}

Передаточная функция ПИД-регулятора:

\begin{equation}
G_c(s) = K_p + \frac{K_i}{s} + K_d s
\label{eq:pid_transfer}
\end{equation}

где $K_p$, $K_i$, $K_d$ — коэффициенты пропорционального, интегрального и дифференциального звеньев соответственно.

\subsection{Алгоритм ПИД-регулятора}

Основные этапы работы ПИД-регулятора представлены в листинге~\ref{lst:pid_controller}.

\begin{verbatim}
import numpy as np
import matplotlib.pyplot as plt

class PIDController:
    def __init__(self, kp=1.0, ki=0.1, kd=0.05, setpoint=0):
        self.kp = kp  # Пропорциональный коэффициент
        self.ki = ki  # Интегральный коэффициент
        self.kd = kd  # Дифференциальный коэффициент
        self.setpoint = setpoint
        
        self.previous_error = 0
        self.integral = 0
        self.output_history = []
        
    def calculate(self, current_value, dt=0.01):
        """Расчет выходного сигнала ПИД-регулятора"""
        error = self.setpoint - current_value
        
        # Пропорциональная составляющая
        proportional = self.kp * error
        
        # Интегральная составляющая
        self.integral += error * dt
        integral_term = self.ki * self.integral
        
        # Дифференциальная составляющая
        derivative = self.kd * (error - self.previous_error) / dt
        
        # Выходной сигнал
        output = proportional + integral_term + derivative
        
        self.previous_error = error
        self.output_history.append(output)
        
        return output
    
    def plot_response(self):
        """Построение графика переходного процесса"""
        plt.figure(figsize=(10, 6))
        plt.plot(self.output_history)
        plt.title('Переходный процесс ПИД-регулятора')
        plt.xlabel('Время')
        plt.ylabel('Выходной сигнал')
        plt.grid(True)
        plt.show()

# Пример использования
pid = PIDController(kp=2.0, ki=0.5, kd=0.1, setpoint=100.0)
current_temperature = 20.0

for i in range(100):
    control_signal = pid.calculate(current_temperature)
    # Симуляция изменения температуры
    current_temperature += control_signal * 0.1
    
print(f"Финальная температура: {current_temperature:.2f} градусов C")
\end{verbatim}

\captionof{figure}{Реализация ПИД-регулятора}
\label{lst:pid_controller}

В листинге \ref{lst:pid_controller} показана реализация ПИД-регулятора с возможностью настройки коэффициентов и визуализации переходного процесса.

\section{Анализ существующих подходов к автоматизации}

В данном разделе рассматриваются современные подходы к автоматизации производственных процессов и их сравнительный анализ.

\subsection{Традиционные методы автоматизации}

К традиционным методам относятся релейно-контактные схемы и простые регуляторы. Основные характеристики:

\begin{itemize}
\item Простота реализации
\item Низкая стоимость
\item Ограниченная функциональность
\item Сложность модификации
\end{itemize}

\subsection{Современные цифровые системы}

Современные системы основаны на микропроцессорной технике и программном обеспечении:

\begin{itemize}
\item Высокая точность управления
\item Гибкость настройки
\item Возможность интеграции с информационными системами
\item Диагностика и мониторинг в реальном времени
\end{itemize}

\subsection{Интеллектуальные системы управления}

Интеллектуальные системы используют методы искусственного интеллекта:

\begin{itemize}
\item Адаптивное управление
\item Самообучение и оптимизация
\item Предсказательная аналитика
\item Автоматическая диагностика неисправностей
\end{itemize}

\section{Методологические основы исследования}

В работе используются следующие методы исследования:

\begin{enumerate}
\item Теоретический анализ существующих решений
\item Математическое моделирование систем управления
\item Компьютерное моделирование и симуляция
\item Экспериментальные исследования на тестовых установках
\item Статистический анализ результатов
\end{enumerate}

\section{Выводы по главе}

В данной главе рассмотрены теоретические основы автоматизации производственных процессов:

\begin{enumerate}
\item Определены ключевые понятия и термины автоматизации
\item Изучена архитектура современных систем автоматизации
\item Рассмотрены математические основы управления технологическими процессами
\item Проведен сравнительный анализ различных методов управления
\item Представлена реализация ПИД-регулятора с возможностью настройки параметров
\item Проанализированы современные подходы к автоматизации
\end{enumerate}

Полученные результаты служат теоретической основой для разработки системы автоматизации в следующих главах работы.
