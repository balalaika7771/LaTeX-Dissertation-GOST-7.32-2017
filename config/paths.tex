% Конфигурация путей к источникам
% По умолчанию источники находятся в папке src/ относительно корня репозитория LaTeX
% Можно переопределить эти пути, изменив значения ниже

% Путь к титульному листу (относительно корня репозитория LaTeX)
% титульный лист: src/title
\newcommand{\titlepath}{src/title}

% Путь к разделу с исполнителями (относительно корня репозитория LaTeX)
% исполнители: src/executors
\newcommand{\executorspath}{src/executors}

% Путь к реферату (относительно корня репозитория LaTeX)
% реферат: src/abstract
\newcommand{\abstractpath}{src/abstract}

% Путь к терминам и определениям (относительно корня репозитория LaTeX)
% термины: src/terms
\newcommand{\termspath}{src/terms}

% Путь к сокращениям и обозначениям (относительно корня репозитория LaTeX)
% сокращения: src/abbreviations
\newcommand{\abbreviationspath}{src/abbreviations}

% Путь к папке с главами (относительно корня репозитория LaTeX)
% главы: src/chapters
\newcommand{\chapterspath}{src/chapters}

% Путь к папке с изображениями (относительно корня репозитория LaTeX)
% изображения: src/images
\newcommand{\imagespath}{src/images}

% Путь к папке с приложениями (относительно корня репозитория LaTeX)
% приложения: src/appendix
\newcommand{\appendixpath}{src/appendix}

% Путь к папке с библиографией (относительно корня репозитория LaTeX)
% библиография: src/bibliography
\newcommand{\bibliographypath}{src/bibliography}

% Команды для удобного подключения файлов
\newcommand{\inputchapter}[1]{\input{\chapterspath/#1}}
\newcommand{\inputappendix}[1]{\input{\appendixpath/#1}}

% Команда для включения изображений с автоматическим путем
% Использование: \includegraphicspath[width=0.8\textwidth]{example_plot.png}
\newcommand{\includegraphicspath}[2][]{%
  \ifx\relax#1\relax
    \includegraphics{\imagespath/#2}%
  \else
    \includegraphics[#1]{\imagespath/#2}%
  \fi
}

