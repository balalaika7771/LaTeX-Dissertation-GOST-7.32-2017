% Настройки страницы по ГОСТ Р 7.0.11-2011 и ГОСТ 7.32-2017
\geometry{
    left=30mm,
    right=15mm,
    top=20mm,
    bottom=20mm
}

% Межстрочный интервал 1.5 по ГОСТу
\onehalfspacing

% Отступ первой строки абзаца 1.25 см по ГОСТу
\setlength{\parindent}{1.25cm}

% Настройки шрифтов по ГОСТу для pdfLaTeX
\usepackage{times}

% Настройки заголовков по ГОСТу 7.32-2017
\titleformat{\chapter}[display]
{\normalfont\bfseries\centering}
{\chaptertitlename\ \thechapter}{20pt}{\Large}

\titleformat{\section}
{\normalfont\bfseries}
{\thesection}{1em}{}

\titleformat{\subsection}
{\normalfont\bfseries}
{\thesubsection}{1em}{}

% Настройки оглавления по ГОСТу 7.32-2017
\renewcommand{\cftchapleader}{\cftdotfill{\cftdotsep}}
\renewcommand{\cftsecleader}{\cftdotfill{\cftdotsep}}
\renewcommand{\cftsubsecleader}{\cftdotfill{\cftdotsep}}

% Настройки отступов в оглавлении по ГОСТ 7.32-2017
\renewcommand{\cftchapindent}{0pt}  % Главы без отступа
\renewcommand{\cftsecindent}{2em}   % Разделы с отступом 2 знака
\renewcommand{\cftsubsecindent}{4em} % Подразделы с отступом 4 знака

% Настройки шрифтов в оглавлении
\renewcommand{\cftchapfont}{\bfseries}
\renewcommand{\cftsecfont}{\normalfont}
\renewcommand{\cftsubsecfont}{\normalfont}

% Настройки отступов между элементами
\setlength{\cftbeforechapskip}{0pt}
\setlength{\cftbeforesecskip}{0pt}
\setlength{\cftbeforesubsecskip}{0pt}

% Настройки библиографии
\addbibresource{bibliography/references.bib}

% Настройки стиля библиографии по ГОСТ 7.32-2017
\DeclareFieldFormat{labelnumberwidth}{#1.}
\setlength{\biblabelsep}{0.5em}
\renewcommand*{\bibfont}{\small}
\setlength{\bibhang}{1.25cm} % абзацный отступ по ГОСТу

% Настройки гиперссылок
\hypersetup{
    colorlinks=true,
    linkcolor=black,
    filecolor=magenta,      
    urlcolor=blue,
    citecolor=black,
    bookmarksnumbered=true,
    bookmarksopen=true,
    pdfstartview=FitH
}

% Настройки листингов кода
\lstset{
    basicstyle=\small\ttfamily,
    breaklines=true,
    frame=single,
    numbers=left,
    numberstyle=\tiny,
    stepnumber=1,
    numbersep=5pt,
    showstringspaces=false,
    showtabs=false,
    tabsize=2,
    captionpos=b,
    commentstyle=\color{gray},
    keywordstyle=\color{blue},
    stringstyle=\color{red}
}

% Настройки нумерации по ГОСТу 7.32-2017
\renewcommand{\thefigure}{\arabic{figure}}
\renewcommand{\thetable}{\arabic{table}}
\renewcommand{\theequation}{\arabic{equation}}

% Настройки подписей к рисункам и таблицам по ГОСТ 7.32-2017
\captionsetup[figure]{
    font=small,
    justification=centering,
    labelsep=endash,
    singlelinecheck=false,
    skip=0pt,  % межстрочный интервал 1.0 (обычный)
    belowskip=0pt,
    aboveskip=0pt
}

\captionsetup[table]{
    font=small,
    justification=raggedright,
    labelsep=endash,
    singlelinecheck=false,
    format=plain
}

% Переопределяем формат подписей для соответствия ГОСТу
\renewcommand{\figurename}{Рисунок}
\renewcommand{\tablename}{Таблица}

% Настройка формата подписи к рисунку по ГОСТ 7.32-2017
\DeclareCaptionFormat{gostfigure}{#1#2 — #3}
\captionsetup[figure]{format=gostfigure}

% Настройки списков рисунков и таблиц
\renewcommand{\listfigurename}{СПИСОК ИЛЛЮСТРАЦИЙ}
\renewcommand{\listtablename}{СПИСОК ТАБЛИЦ}

% Настройки приложений
\renewcommand{\appendixname}{Приложение}
\renewcommand{\appendixpagename}{Приложения}
