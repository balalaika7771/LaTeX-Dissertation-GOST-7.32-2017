% Настройки страницы по ГОСТ Р 7.0.11-2011 и ГОСТ 7.32-2017
\geometry{
    left=30mm,
    right=15mm,
    top=20mm,
    bottom=20mm
}

% Настройки нумерации страниц по требованиям РУДН
% Нумерация внизу/вверху посередине листа
\pagestyle{plain}
\setlength{\footskip}{20pt}

% Межстрочный интервал 1.5 по ГОСТу
\onehalfspacing

% Отступ первой строки абзаца 1.25 см по ГОСТу
\setlength{\parindent}{1.25cm}

% Настройки шрифтов по ГОСТу для pdfLaTeX
\usepackage{times}

% Настройки заголовков по ГОСТу 7.32-2017 и требованиям РУДН
% Заголовки структурных элементов - по центру, прописными буквами, без точки в конце
\titleformat{\chapter}[display]
{\normalfont\bfseries\centering}
{\MakeUppercase{\chaptertitlename\ \thechapter}}{20pt}{\Large}

\titleformat{\section}
{\normalfont\bfseries\centering}
{\MakeUppercase{\thesection}}{1em}{\MakeUppercase}

\titleformat{\subsection}
{\normalfont\bfseries\centering}
{\MakeUppercase{\thesubsection}}{1em}{\MakeUppercase}

% Настройки оглавления по ГОСТу 7.32-2017
\renewcommand{\cftchapleader}{\cftdotfill{\cftdotsep}}
\renewcommand{\cftsecleader}{\cftdotfill{\cftdotsep}}
\renewcommand{\cftsubsecleader}{\cftdotfill{\cftdotsep}}

% Настройки отступов в оглавлении по ГОСТ 7.32-2017
\renewcommand{\cftchapindent}{0pt}  % Главы без отступа
\renewcommand{\cftsecindent}{2em}   % Разделы с отступом 2 знака
\renewcommand{\cftsubsecindent}{4em} % Подразделы с отступом 4 знака

% Настройки шрифтов в оглавлении
\renewcommand{\cftchapfont}{\bfseries}
\renewcommand{\cftsecfont}{\normalfont}
\renewcommand{\cftsubsecfont}{\normalfont}

% Настройки отступов между элементами
\setlength{\cftbeforechapskip}{0pt}
\setlength{\cftbeforesecskip}{0pt}
\setlength{\cftbeforesubsecskip}{0pt}

% Настройки библиографии
\addbibresource{bibliography/references.bib}

% Настройки стиля библиографии по ГОСТ 7.32-2017 и требованиям РУДН
\DeclareFieldFormat{labelnumberwidth}{#1.}
\setlength{\biblabelsep}{0.5em}
\renewcommand*{\bibfont}{\small}
\setlength{\bibhang}{1.25cm} % абзацный отступ по ГОСТу

% Формат ссылок [номер, с.страница] по требованиям РУДН
% Используем стандартный формат gost-numeric с настройками
\DeclareFieldFormat{cite}{[#1]}
\DeclareFieldFormat{textcite}{[#1]}

% Настройки гиперссылок
\hypersetup{
    colorlinks=true,
    linkcolor=black,
    filecolor=magenta,      
    urlcolor=blue,
    citecolor=black,
    bookmarksnumbered=true,
    bookmarksopen=true,
    pdfstartview=FitH
}

% Настройки нумерации по ГОСТу 7.32-2017
\renewcommand{\thefigure}{\arabic{figure}}
\renewcommand{\thetable}{\arabic{table}}
\renewcommand{\theequation}{\arabic{equation}}

% Настройки подписей к рисункам и таблицам по ГОСТ 7.32-2017
\captionsetup[figure]{
    font=small,
    justification=centering,
    singlelinecheck=false,
    skip=0pt,
    belowskip=0pt,
    aboveskip=0pt,
    hypcap=false
}

\captionsetup[table]{
    font=small,
    justification=raggedright,
    singlelinecheck=false,
    skip=0pt,
    belowskip=0pt,
    aboveskip=0pt,
    hypcap=false
}

% Переопределяем формат подписей для соответствия ГОСТу
\renewcommand{\figurename}{Рисунок}
\renewcommand{\tablename}{Таблица}

% Настройка формата подписи к рисунку по ГОСТ 7.32-2017
\DeclareCaptionFormat{gostfigure}{#1#2 — #3}
\captionsetup[figure]{format=gostfigure}

% Настройка формата подписи к таблице по ГОСТ 7.32-2017
\DeclareCaptionFormat{gosttable}{#1#2 — #3}
\captionsetup[table]{format=gosttable}

% Настройка переноса таблиц с заголовком "Продолжение таблицы X"
\usepackage{longtable}
\usepackage{ltxtable}

% Команда для создания продолжения таблицы
\newcommand{\tablecontinuation}[1]{%
    \multicolumn{1}{c}{\textbf{Продолжение таблицы #1}}%
}

% Настройка переноса листингов кода
\usepackage{fancyvrb}
\usepackage{listings}

% Настройки для листингов кода (единый стиль)
\lstset{
    breaklines=true,
    breakatwhitespace=true,
    breakautoindent=true,
    postbreak=\mbox{\textcolor{red}{$\hookrightarrow$}\space},
    frame=single,
    numbers=left,
    numberstyle=\tiny,
    stepnumber=1,
    numbersep=5pt,
    showstringspaces=false,
    showtabs=false,
    tabsize=2,
    basicstyle=\small\ttfamily,
    keywordstyle=\color{blue}\bfseries,
    commentstyle=\color{green!60!black},
    stringstyle=\color{red},
    extendedchars=true,
    inputencoding=utf8,
    upquote=true,
    captionpos=t,
    abovecaptionskip=10pt,
    belowcaptionskip=10pt,
    literate={а}{{\cyra}}1 {б}{{\cyrb}}1 {в}{{\cyrv}}1 {г}{{\cyrg}}1 {д}{{\cyrd}}1 {е}{{\cyre}}1 {ё}{{\cyryo}}1 {ж}{{\cyrzh}}1 {з}{{\cyrz}}1 {и}{{\cyri}}1 {й}{{\cyrishrt}}1 {к}{{\cyrk}}1 {л}{{\cyrl}}1 {м}{{\cyrm}}1 {н}{{\cyrn}}1 {о}{{\cyro}}1 {п}{{\cyrp}}1 {р}{{\cyrr}}1 {с}{{\cyrs}}1 {т}{{\cyrt}}1 {у}{{\cyru}}1 {ф}{{\cyrf}}1 {х}{{\cyrh}}1 {ц}{{\cyrc}}1 {ч}{{\cyrch}}1 {ш}{{\cyrsh}}1 {щ}{{\cyrshch}}1 {ъ}{{\cyrhrdsn}}1 {ы}{{\cyry}}1 {ь}{{\cyrsftsn}}1 {э}{{\cyrerev}}1 {ю}{{\cyryu}}1 {я}{{\cyrya}}1
}

% Настройки для подписей листингов (как у таблиц - слева с тире)
\captionsetup[lstlisting]{
    format=hang,
    justification=raggedright,
    singlelinecheck=false,
    font=small,
    labelfont=bf,
    textfont=normalfont,
    skip=10pt,
    position=top,
    labelsep=endash
}

% Окружение для кода: \begin{CodeBlock}{ЯЗЫК}{Описание}{lst:метка} ... \end{CodeBlock}
\lstnewenvironment{CodeBlock}[3]{%
    \lstset{%
        language=#1,
        caption={#2},
        label={#3}
    }%
}{%
}

% Настройки примечаний согласно ГОСТ 7.32-2017
% Единый способ: окружение notes
\newenvironment{notes}{%
    \par\noindent\textbf{Примечания:}%
    \begin{enumerate}%
}{%
    \end{enumerate}%
}

% Настройка межстрочного интервала для многострочных подписей
\captionsetup[table]{format=gosttable,skip=6pt}

% Настройки списков рисунков и таблиц
\renewcommand{\listfigurename}{СПИСОК ИЛЛЮСТРАЦИЙ}
\renewcommand{\listtablename}{СПИСОК ТАБЛИЦ}

% Настройки приложений
\renewcommand{\appendixname}{Приложение}
\renewcommand{\appendixpagename}{Приложения}

% ===== АВТОМАТИЧЕСКИЕ ПРАВИЛА ГОСТ =====

% Автоматические переносы слов по ГОСТу
\usepackage[english,russian]{babel}
\babelprovide[import]{russian}
\babelprovide[import]{english}

% Автоматические отступы для списков по ГОСТу
\setlength{\leftmargini}{2.5em}  % отступ для enumerate
\setlength{\leftmarginii}{2em}   % отступ для вложенных списков
\setlength{\leftmarginiii}{1.5em}
\setlength{\leftmarginiv}{1em}

% Автоматическое форматирование списков
\renewcommand{\labelenumi}{\arabic{enumi})}
\renewcommand{\labelenumii}{\arabic{enumi}.\arabic{enumii})}
\renewcommand{\labelenumiii}{\arabic{enumi}.\arabic{enumii}.\arabic{enumiii})}
\renewcommand{\labelenumiv}{\arabic{enumi}.\arabic{enumii}.\arabic{enumiii}.\arabic{enumiv})}

% Автоматические отступы для itemize
\renewcommand{\labelitemi}{---}
\renewcommand{\labelitemii}{--}
\renewcommand{\labelitemiii}{-}
\renewcommand{\labelitemiv}{.}

% Автоматическое форматирование формул по ГОСТу
\renewcommand{\theequation}{\arabic{equation}}
\setlength{\abovedisplayskip}{6pt}
\setlength{\belowdisplayskip}{6pt}
\setlength{\abovedisplayshortskip}{0pt}
\setlength{\belowdisplayshortskip}{0pt}

% Автоматические правила для ссылок
\renewcommand{\autoref}[1]{\textbf{\ref{#1}}}
\renewcommand{\nameref}[1]{\textbf{\ref{#1}}}

% Автоматические правила для абзацев
\setlength{\parskip}{0pt}  % без отступов между абзацами
\setlength{\parindent}{1.25cm}  % отступ первой строки по ГОСТу

% Автоматические правила для переноса страниц
\widowpenalty=10000
\clubpenalty=10000
\raggedbottom

% Автоматические правила для переноса слов в тексте
\hyphenpenalty=50
\exhyphenpenalty=50
\doublehyphendemerits=10000
\finalhyphendemerits=5000

% Автоматические правила для таблиц
\setlength{\tabcolsep}{6pt}
\setlength{\arrayrulewidth}{0.4pt}
\renewcommand{\arraystretch}{1.2}

% Автоматические правила для рисунков
\setlength{\floatsep}{12pt}
\setlength{\textfloatsep}{12pt}
\setlength{\intextsep}{12pt}

% Автоматические правила для списков литературы
\renewcommand{\bibname}{СПИСОК ИСПОЛЬЗОВАННЫХ ИСТОЧНИКОВ}

% Автоматические правила для оглавления
\renewcommand{\contentsname}{ОГЛАВЛЕНИЕ}

% Автоматические правила для реферата
\renewcommand{\abstractname}{РЕФЕРАТ}

% Автоматические правила для заключения
\providecommand{\conclusionname}{ЗАКЛЮЧЕНИЕ}

% Автоматические правила для введения
\providecommand{\introductionname}{ВВЕДЕНИЕ}

% ===== КОМАНДЫ ДЛЯ УПРОЩЕНИЯ НАПИСАНИЯ =====

% Создание рисунка с правильным форматированием
\newcommand{\autofigure}[4]{%
    \begin{figure}[H]%
        \centering%
        \includegraphics[width=#2]{#1}%
        \caption{#3}%
        \label{#4}%
    \end{figure}%
}

% Создание таблицы с правильным форматированием
\newcommand{\autotable}[3]{%
    \begin{table}[H]%
        \centering%
        \caption{#2}%
        \label{#3}%
        #1%
    \end{table}%
}

% Создание формулы с правильным форматированием
\newcommand{\autoformula}[2]{%
    \begin{equation}%
        #1%
        \label{#2}%
    \end{equation}%
}

% Создание списка с правильным форматированием
\newcommand{\autolist}[1]{%
    \begin{enumerate}%
        #1%
    \end{enumerate}%
}

% Создание маркированного списка
\newcommand{\autobulletlist}[1]{%
    \begin{itemize}%
        #1%
    \end{itemize}%
}

% Создание ссылки на рисунок
\newcommand{\autofigref}[1]{%
    \textbf{рисунок \ref{#1}}%
}

% Создание ссылки на таблицу
\newcommand{\autotableref}[1]{%
    \textbf{таблица \ref{#1}}%
}

% Создание ссылки на формулу
\newcommand{\autoformularef}[1]{%
    \textbf{формула \ref{#1}}%
}

% Создание абзаца с правильным отступом
\newcommand{\autopar}[1]{%
    \par\noindent#1%
}

% ===== КОМАНДЫ ДЛЯ АВТОМАТИКИ (ПО ГОСТ 7.32-2017) =====

% Стили для блок-схем по ГОСТ 19.701-90
\tikzstyle{startstop} = [rectangle, rounded corners, minimum width=3cm, minimum height=1cm, text centered, draw=black, fill=red!30]
\tikzstyle{process} = [rectangle, minimum width=3cm, minimum height=1cm, text centered, draw=black, fill=orange!30]
\tikzstyle{decision} = [diamond, minimum width=3cm, minimum height=1cm, text centered, draw=black, fill=yellow!30]
\tikzstyle{io} = [trapezium, trapezium left angle=70, trapezium right angle=110, minimum width=3cm, minimum height=1cm, text centered, draw=black, fill=blue!30]
\tikzstyle{arrow} = [thick,->,>=stealth]

% Блок-схема алгоритма по ГОСТ 19.701-90
\newcommand{\autoblockdiagram}[3]{%
    \begin{figure}[H]%
        \centering%
        \begin{tikzpicture}[node distance=1.5cm]%
            \node (start) [startstop] {Начало};%
            \node (input) [io, below of=start] {Ввод данных};%
            \node (process) [process, below of=input] {Обработка};%
            \node (decision) [decision, below of=process] {Условие?};%
            \node (output) [io, below of=decision] {Вывод результата};%
            \node (stop) [startstop, below of=output] {Конец};%
            %
            \draw [arrow] (start) -- (input);%
            \draw [arrow] (input) -- (process);%
            \draw [arrow] (process) -- (decision);%
            \draw [arrow] (decision) -- node[anchor=east] {Да} (output);%
            \draw [arrow] (decision) -- node[anchor=south] {Нет} ++(-3,0) |- (process);%
            \draw [arrow] (output) -- (stop);%
        \end{tikzpicture}%
        \caption{#2}%
        \label{#3}%
    \end{figure}%
}

% Схема системы управления по ГОСТ 2.701-2008
\newcommand{\autocontrolscheme}[3]{%
    \begin{figure}[H]%
        \centering%
        \begin{tikzpicture}[node distance=2cm]%
            \node (sensor) [rectangle, draw=black, fill=blue!20, minimum width=2cm, minimum height=1cm] {Датчик};%
            \node (controller) [rectangle, draw=black, fill=green!20, minimum width=2cm, minimum height=1cm, right of=sensor] {Контроллер};%
            \node (actuator) [rectangle, draw=black, fill=red!20, minimum width=2cm, minimum height=1cm, right of=controller] {Исполнительный механизм};%
            \node (process) [rectangle, draw=black, fill=yellow!20, minimum width=2cm, minimum height=1cm, below of=controller] {Объект управления};%
            %
            \draw [arrow] (sensor) -- node[above] {Измерение} (controller);%
            \draw [arrow] (controller) -- node[above] {Управление} (actuator);%
            \draw [arrow] (actuator) -- node[right] {Воздействие} (process);%
            \draw [arrow] (process) -- node[left] {Состояние} (sensor);%
        \end{tikzpicture}%
        \caption{#2}%
        \label{#3}%
    \end{figure}%
}

% Временная диаграмма
\newcommand{\autotimingdiagram}[3]{%
    \begin{figure}[H]%
        \centering%
        \begin{tikzpicture}[scale=0.8]%
            \draw[thick] (0,0) -- (8,0) node[right] {Время};%
            \draw[thick] (0,0) -- (0,4) node[above] {Сигналы};%
            %
            \draw[blue,thick] (0,3) -- (2,3) -- (2,1) -- (4,1) -- (4,3) -- (6,3) -- (6,1) -- (8,1);%
            \node[blue,left] at (0,3) {Сигнал 1};%
            %
            \draw[red,thick] (0,2) -- (1,2) -- (1,0.5) -- (3,0.5) -- (3,2) -- (5,2) -- (5,0.5) -- (8,0.5);%
            \node[red,left] at (0,2) {Сигнал 2};%
            %
            \draw[green,thick] (0,1) -- (1.5,1) -- (1.5,0.2) -- (3.5,0.2) -- (3.5,1) -- (5.5,1) -- (5.5,0.2) -- (8,0.2);%
            \node[green,left] at (0,1) {Сигнал 3};%
        \end{tikzpicture}%
        \caption{#2}%
        \label{#3}%
    \end{figure}%
}

% Структурная схема
\newcommand{\autostructure}[3]{%
    \begin{figure}[H]%
        \centering%
        \begin{tikzpicture}[node distance=2cm]%
            \node (input) [rectangle, draw=black, fill=blue!20] {Входной сигнал};%
            \node (block1) [rectangle, draw=black, fill=green!20, right of=input] {Блок 1};%
            \node (block2) [rectangle, draw=black, fill=orange!20, right of=block1] {Блок 2};%
            \node (output) [rectangle, draw=black, fill=red!20, right of=block2] {Выходной сигнал};%
            %
            \draw [arrow] (input) -- (block1);%
            \draw [arrow] (block1) -- (block2);%
            \draw [arrow] (block2) -- (output);%
        \end{tikzpicture}%
        \caption{#2}%
        \label{#3}%
    \end{figure}%
}

% Функциональная схема
\newcommand{\autofunctional}[3]{%
    \begin{figure}[H]%
        \centering%
        \begin{tikzpicture}[node distance=1.5cm]%
            \node (input) [circle, draw=black, fill=blue!20] {Вход};%
            \node (process1) [rectangle, draw=black, fill=green!20, right of=input] {Обработка 1};%
            \node (process2) [rectangle, draw=black, fill=orange!20, below of=process1] {Обработка 2};%
            \node (output) [circle, draw=black, fill=red!20, right of=process2] {Выход};%
            %
            \draw [arrow] (input) -- (process1);%
            \draw [arrow] (process1) -- (process2);%
            \draw [arrow] (process2) -- (output);%
        \end{tikzpicture}%
        \caption{#2}%
        \label{#3}%
    \end{figure}%
}

% Принципиальная схема
\newcommand{\autoschematic}[3]{%
    \begin{figure}[H]%
        \centering%
        \begin{circuitikz}[scale=0.8]%
            \draw (0,0) to[resistor, l=$R_1$] (2,0) to[capacitor, l=$C_1$] (4,0) to[inductor, l=$L_1$] (6,0);%
            \draw (0,-2) to[voltage source, l=$V_{in}$] (0,0);%
            \draw (6,0) to[resistor, l=$R_2$] (6,-2) to[ground] (0,-2);%
        \end{circuitikz}%
        \caption{#2}%
        \label{#3}%
    \end{figure}%
}